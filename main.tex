\documentclass[a4j,11pt]{jarticle}
\usepackage[dvipdfmx]{graphicx}

\author{藤村 忠 @chu\_fuzimura}
\title{パイドン 読書会資料}
\date{2019/3/30}

\begin{document}
\maketitle

\section*{課題本}
\begin{itemize}
    \item[書名:] ソークラテースの弁明・クリトーン・パイドーン
    \item[フォーマット:] Kindle
    \item[出版社:] 新潮社 (1968/6/30)  
\end{itemize}

引用の()内の数字は kindle No. を表す。

\section{哲学者は死ぬことを求める}

\subsection{主張}
真に哲学にたずさわる人々は 、ただひたすら死ぬこと 、死を全うすることを目ざしている

\subsection{根拠}

\begin{itemize}
    \item 死は肉体からの魂の離脱である
    \item 本物の哲学者は肉体的快楽を軽蔑する
    \item 哲学者の興味は肉体ではなく魂に向いている
    \item 五感は厳密でも確実でもない
    \item 真実在(正しさや美や善そのもの)が存在する
    \item 真実在は身体感覚によって捉えられない
    \item 真実在は思惟のみによって捉えられる
\end{itemize}

\begin{quotation}
    思惟が最も見事に働くのは 、魂が聴覚 、視覚 、苦痛 、快楽といった肉体的なものにわずらわされることなく 、肉体を離れて 、できるだけ魂だけになって 、肉体との協力も接触も能うかぎりこばみ 、ものの真実を追求するときなのだ。(1666)
    
    もし何かを純粋に見ようとするなら 、肉体から離れて 、魂そのものによって 、物そのものを見なければならないということは 、われわれには確かに明白な事実なのだ。(1698)
\end{quotation}

\subsection{導出}
前提として、死は肉体からの魂の離脱であることを認める。また美そのものや等さそのものといった真実在が存在することを認める。美そのものとは私たちが現実にある美しいものを身体感覚によって、肉体によって捉えることによって美を感じ、美そのものを感じる。

あるものを美しいと思う人もいれば、美しくないと思うひともいる。しかし美の真実在は全ての人にとって美そのものとして存在している。身体感覚によって捉えられる個別の美は不完全であることがここから導ける。

また真実在は身体感覚ではなく思惟のみによって捉えられるということもわかる。

また、魂は不滅である。魂は不滅である、かつ肉体が真実在への認識を阻害する、かつ哲学者は真実在を捉えようとし続けている。このことから哲学者は死を求めるということが証明できる。

\section{魂は不滅である}

\subsection{主張}
魂は不滅である。

\subsection{ケベスの反論}
\begin{quotation}
    魂は肉体を離れると 、もうどこにも存在せず 、人間が死んだその日に滅びて 、なくなってしまうのではないか 、肉体から抜出すとすぐに 、まるで息か煙のように飛び散り消え去って 、もうどこにもあとかたもなくなってしまうのではないかと(1784)
    
    人間が死んでもその魂はなお存在し、なんらかの力と知恵とを持ちつづけるということは 、おそらく少なからぬ説得と証明を必要といたします。(1790)
\end{quotation}

\subsection{導出}
魂が不滅であることは以下の2つの命題を証明する必要がある。

\begin{itemize}    
    \item 魂は生前も存在している
    \item 魂は死後も消滅しない
\end{itemize}

あらゆるものは反対の性質から生じる。例えば熱いものは冷たいものから生じる。生の反対は死である。よって死は生から、生は死から生じる。
このことから生まれ変わりが存在することが証明できる。生が死から生じるというのは生まれ変わりである。生まれ変わるがあるということは、生まれる以前も魂はどこかに存在していたということである。よって魂は生前も存在している。

\section{想起説}
\subsection{主張}
学びは想起である。

\subsection{根拠}
\begin{itemize}
    
    \item なにかを想起するにはそれをいつか以前にしっていたのでなければならない。
    \item 人は身体感覚でなにかをとらえるとき、それそのものだけではなく別のものを想起する
    \item 想起は似ているものによって起こる場合とそうでないものによって起こる場合がある。
    \item 知覚されたものは想起されたものと比較して欠落がある。
    \item 等さそのもの(真実在)がある
    \item 等さそのものがなんであるかを知っている。それは個別の事象、なにかとなにかをくらべて等しいのを見ることによって等さを知る。
    \item 個別の等さはひとによってちがう。ある人にとっては等しい石も、別のひとにとってはそう感じないかもしれない。しかし等さそのものはみんな同じものをかんがえている。
    \item 等しい事物と等さそのものは同じではない。
    \item 等しい事物から等さを知った、これは想起である。
    
\end{itemize}

\begin{quote}
    もし誰かが何かを見て 、いま自分が見ているものはほかのあるものに似たものでありたいと願ってはいるけれども欠けていて 、そのあるものそっくりであることができず劣っているのだ 、と考えるとすれば 、彼は似てはいるけれども欠けているという 、その規準になるものを 、前に見たことがあるのでなければならない。(1930)
\end{quote}

\subsection{導出}

魂が生前も存在していたということを想起説から証明する。

まず想起するためには以前にそれを知っていたのでなければならない。人は身体感覚でなにかを捉えるとき、それそのものだけではなく別のものを想起する。たとえば服を見て、それの所有者を想起するように。

私たちは個別の等さをみて、等さそのものを認識することができる。これは想起である。等さそのものは身体感覚によっては知覚できない。しかし想起したからには、私たちは等さそのものをどこかで知っていたのでなければならない。それは生まれる以前である。よって生前も魂が存在する。

\section{魂は消滅しない}
\subsection{主張}
魂は肉体の死後も消滅せず、不滅である。

\subsection{根拠}    
\begin{itemize}
    \item 合成物は分解する。
    \item 合成物でないものは分解できない
    \item 同一で変化しないものが非合成物
    \item 変化するものが合成物
    \item 真実在は同一である
    \item 不可視的なものは常に同一である
    \item 可視的なものは同一ではない
    \item 肉体は可視的である
    \item 魂は不可視的である
    \item 肉体は分解する
    \item 魂は分解しない
\end{itemize}

\subsection{導出}
魂が肉体の死後も消滅しないということについて、消滅の性質から考える。
合成物は分解する。合成物でないものはもうそれ以上分解することができない。合成物は分解するがゆえに変化し、非合成物は分解しないがゆえに普遍である。真実在は常に同一である。真実在は非合成物である。

不可視なものは常に同一である。よって不可視なものは非合成物である。可視なものは変化する。ゆえに可視なものは合成物である。

魂は不可視である。よって魂は非合成物である。よって魂は分解しないので不滅である。

\section{魂は調和ではない}
\subsection{シミアスの主張}
\begin{itemize}
    \item 魂の不滅についての議論は琴の調和にもあてはまる。
    \item 調和は不可視である
    \item 琴は可視である。
    \item 魂は調和である。
    \item 壊れた琴が調和を生み出せないように死んだ肉体は魂を生み出せない
\end{itemize}

\begin{quote}    
    つまりわれわれの肉体は、熱冷乾湿その他同様の相反する性質のあいだの緊張関係によって結合が保たれており、われわれの魂は、これらの要素そのものがたがいに正しい適当な割合でまぜ合わされる場合に生じる混合であり、調和である、と考えられます。(2259)
\end{quote}

\section{生成と消滅}

\section{真実在は存在する}

https://github.com/paragraph14/Phaedo

\end{document}